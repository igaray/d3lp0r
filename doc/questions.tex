TL;DR section: (short answers for all questions)

== Introduction 1. What was the motivation to participate in the contest?  The
group's main motivation was to apply argumentation via defeasible logic
programming (DeLP) in a multi-agent gaming situation and to test the
integration of the different technologies used.

2. What is the history of the team?  The LIDIA research group was established
in 1992 in the Universidad Nacional del Sur, and it is the first time our time
participates in the contest.

3. What is the name of your team?  The team's name is d3lp0r.

4. How many developers and designers did you have?  At what level of education
are your team members?  The d3lp0r team was formed incorporating 6 graduate
students, 2 Ph.D. students and 3 professors.

5. From which field of research do you come form?  Which work is related?  The
LIDIA research group has been working in Artificial Intelligence and
Argumentation via Defeasible Logic Programming for almost 20 years now, and
the DeLP server technology developed has been used in the contest.

== System Analysis and Design 1. If some multi-agent system methodology such
as Prometheus, O-MaSE, or Tropos was used, how did you use it? If you did not
what were the reasons?  No design methodology specific to multi-agent systems
was used. However, development was conducted using a simplified XP (extreme
programming) methodology. 

2. Is the solution based on the centralisation of coordination/information on
a specific agent? Conversely if you plan a decentralised solution, which
strategy do you plan to use?  The solution follows a decentralised
architecture in which agents run completely decoupled in different processes
while sharing nothing.

3. What is the communication strategy and how complex is it?  Percepts are
communicated among agent members of the team via a broadcast mechanism
developed as part of the multi-agent system. This design was chosen for its
minimal complexity.

4. How are the following agent features considered/implemented: autonomy,
proactiveness, reactiveness?  Agents are completely autonomous;
decision-making takes place individually at the agent level, with no
intervention from human operators or a central intelligence agency within the
system.  Agents assign priorities to different possible goals depending on
their desires, and plan in order to achieve the most valuable goal. This
results in a less reactive and more autonomous way in which an agent acts.


5. Is the team a truly multi-agent system or rather a centralised system in
disguise?  The team is a truly multi-agent system, and has absolutely no
centralised characteristics.


6. How much time (man hours) have you invested (approximately) for
implementing your team?  I don't know, but IT'S OVER 9.000!


7. Did you discuss the design and strategies of your agent team with other
developers? To which extent did your test your agents playing with other
teams?  Experience from a previous instance of the MAPC was shared with our
teams by members of the ARGONAUTS team from Universität Dortmund. Although the
initial plan was to run tests against other agent teams prior to the
competition, time constraints made this impossible.


== Software Architecture 1. Which programming language did you use to
implement the multi-agent system?  The agent system was implemented using
Python 2.7 and SWI Prolog 5.10.5. DeLP, a defeasible logic language, was used
as a service within Prolog.

2. Did you use multi-agent programming languages? Why or why not to use a
multi-agent programming language?  No multi-agent programming
languages/patforms/frameworks were used due to previous experience indicating
a general lack of flexibility, and a lack of familiarity on behalf of the
development team.

3. How have you mapped the designed architecture (both multi-agent and
individual agent architectures) to programming codes i.e., how did you
implement specific agent-oriented concepts and designed artifacts using the
programming language?  The perception is processed by the Python program, that
parses the XML. Then, it sends it to the Percept Server that every step merges
all perceptions, and delivers them back to the agents.  The Python code
asserts all the perception into Prolog, then querying it for the next action
to be executed.  Prolog handles all the decision making, argumentation and
planning, and returns the action binded to a variable to Python, that then
generates with it an XML to be sent to the server.

4. Which development platforms and tools are used? How much time did you
invest in learning those?  Both Linux and the Windows operating system were
used as development platforms, since the language runtimes chosen for
implementation were portable.  We used git as our revision control system. In
general, we did not spend much time in learning it, since some of us had
already worked with it.

5. Which runtime platforms and tools (e.g. Jade, AgentScape, simply Java,
....) are used? How much time did you invest in learning those?  Python and
Prolog were the chosen languages for the development of the system. Most of us
had already worked with both of them, so we did not spend much time learning
those.

6. What features were missing in your language choice that would have
facilitated your development task?


7. What features of your programming language has simplified your development
task?  Python's amenity to rapid application development and
'batteries-included philosophy' facilitated implementing the communication
layer to the MASSimg server, parsing of peceptions, rapid addition of planned
features and bug correction.  We made use of Prolog's declarative nature to
model states of the world, and it also made it more straightforward to
implement search algorithms.


8. Which algorithms are used/implemented?  Search algorithms, as Uniform Cost
Search and Depth First Search, as well as the zone-coloring algorithm were
implemented in Prolog.  The implementation of Defeasible Logic Programming
(DeLP) by the LIDIA was used for the deliberative process.

9. How did you distribute the agents on several machines? And if you did not
please justify why.  Initial plans were to distribute agents on several
machines. Each agents runs as a separate process, and communicates with others
via TCP sockets. After some experience and bencmarking, agents were run on one
machine due to performance issues. Having the choice was a benefit of the
proposed design.

10. To which extend is the reasoning of your agents synchronized with the
receive-percepts/send-action cycle?


11. What part of the development was most difficult/complex? What kind of
problems have you found and how are they solved?  The most difficult problems
were related to optimization. Much of our time has been spent in reducing the
complexity of our algorithms, and the times they are called.

12. How many lines of code did you write for your software?  Total LOC is
5842.

== Stategies, Details, and Statistics
1. What is the main strategy of your
team?  The main strategy of the team consists of detecting profitable zones
from the explored vertices, and positioning the agents correctly to maintain,
defend and expand the zones.

2. How does the overall team work together? (coordination, information
sharing, ...) On each perceive/act cycle, agents receive the percept from the
MASSim server, separate the information which will remain private and which
will be shared.  The public part of the percept is sent to the percept server,
which performs a union of all percept and send the difference back to each
agent. After receiving the joint percept, the agents enter a belief setting
phase, and later an argumentation phase.


3. How do your agents analyze the topology of the map? And how do they exploit
their findings?  Agents make no assumption about the map topology. They will
prefer higher valued nodes over lower ones.

4. How do your agents communicate with the server?  Some functionality
provided by the eismassim library was reimplemented in a connection library in
Python.

5. How do you implement the roles of the agents? Which strategies do the
different roles implement?  Agents recover their assigned role from the
simulation start message.  (????????????????????????)

6. How do you find good zones? How do you estimate the value of zones?  If an
agent is not being part of any zone, it tries to regroup with a partner.  When
a zone is formed, and the agent is part of it, for each potentally beneficial
neighbor node, the agent calculates how much points would they win if it
moves, and that information is used by the decision taking module.

7. How do you conquer zones? How do you defend zones if attacked? Do you
attack zones?  Both attacking and defense are implicitly implemented.
Sabouteurs attack enemies that are near, so they might attack them if they
enter our team's zone, as well as when they are in their zone. Any other agent
of another role can go to a node that has, for example, two agents, one for
each team, in order to expand the zone, occupying the contested node, and
implicitly defending it.


8. Can your agents change their behavior during runtime? If so, what triggers
the changes?  Our agents do not change their behavior during runtime. It is
actually very easy to add this feature, but we had not enough time to
implement this.


9. What algorithm(s) do you use for agent path planning?  Path planning is
implemented with an Uniform Cost Search. What we tried to minimize was the
amount of steps required to achieve the goal, rather than the spent energy.
The returned result is the list of actions to be done.

10. How do you make use of the buying-mechanism?  Agents followed a list of
predefined buying actions, when the necessary amount of money was reached.

11. How important are achievements for your overall strategy?  Our agents did
not have achievements in consideration. However, they managed to achieve a
significant number of them, since this behavior was implicitly implemented.


12. Do your agents have an explicit mental state?

13. How do your agents communicate? And what do they communicate?  Agents only
communicate their perceptions via a perception server implemented in Python.

14. How do you organize your agents? Do you use e.g. hierarchies? Is your
organization implicit or explicit?

15. Is most of your agents’ behavior emergent on and individual and team
level?

16. If your agents perform some planning, how many steps do they plan ahead?
The agents make plans as long as the selected intention requieres. This may
sound excesive, but the possible goals were previously selected for their
potential, taking in consideration their distance (in nodes, not in actions).
However, plans are recalculated in every step.

== Conclusion

1. What have you learned from the participation in the contest?
    Being our first experience building a system this size, we learned several lessons about working in big projects, such as setting standards and synchronizing versions of the technologies used.


2. Which are the strong and weak points of the team?
    Our team is formed by a group of friends, so a strong point in the
    developing proccess was the cohesion and comradeship. We had
    already worked together, both in university's projects and in
    small freelance projects, so we well knew what to expect from each other,
    and each member's capabilities.

  
3. How suitable was the chosen programming language, methodology, tools, and
algorithms?
    We are really happy with our decisions involving programming languages,
    tools and algorithms. Of course, we had many problems, and much time was
    spent deciding what to choose, but having all the proccess in
    consideration, we may had took the right calls.

4. What can be improved in the context for next year?
    There were several hotfixes that were written and deployed at the same time we were facing our competitors due to the lack of testing in the actual context of the competition. This situation should obviously not happen, and adding much more real testing is one of our main priorities for next year's competition.

5. Why did your team perform as it did? Why did the other teams perform
better/worse than you did?

6. Which other research fields might be interested in the Multi-Agent
Programming Contest?


7. How can the current scenario be optimized? How would those optimization pay
off?

    More information for the nodes, including something useful for a directed search (i.e., absolute coordinates), would help in
    the implementation of a A* search (which would decrease execution time). Defining the most valuable zones randomly would benefit teams that thoughtfully look for and conquer good zones, rather than teams that assume that the center of the map is the most valuable zone and don't explore the rest.

    A more informing/informational/apprising? feedback from the server would
    be appreciated, specially involving errors.

    Finally, we think it will be really helpful for all that we had test
    matches in a more early stage, in order to have more time to correct
    errors in the client.
