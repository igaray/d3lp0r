%TL;DR section: (short answers for all questions)

\section{Short Answers}

% == Introduction 1. 

\subsection{Introduction}

\begin{question}
What was the motivation to participate in the contest?  
\end{question}

The group's main motivation was to apply argumentation via defeasible logic
programming (DeLP) in a multi-agent gaming situation and to test the integration
of the different technologies used.

\begin{question}
What is the history of the team?  
\end{question}

The LIDIA research group was established in 1992 in the Universidad Nacional del
Sur, and it is the first time our team participates in the contest.

\begin{question}
What is the name of your team?  
\end{question}

The team's name is d3lp0r.

\begin{question}
How many developers and designers did you have?  At what level of education are
your team members?  
\end{question}

The d3lp0r team was formed incorporating six graduate students, two Ph.D.
students and three professors.

\begin{question}
From which field of research do you come form?  Which work is related?  
\end{question}

The LIDIA research group has been working in Artificial Intelligence and
Argumentation via Defeasible Logic Programming for almost 20 years now, and the
DeLP server technology developed has been used in the contest.

\subsection{System Analysis and Design}
\setcounter{question}{0}

% == System Analysis and Design
\begin{question}
If some multi-agent system methodology such as Prometheus, O-MaSE, or Tropos was
used, how did you use it? If you did not what were the reasons?  
\end{question}

No design methodology specific to multi-agent systems was used. However,
development was conducted using a simplified XP (Extreme Programming)
methodology. 

\begin{question}
Is the solution based on the centralisation of coordination/information on
a specific agent? Conversely if you plan a decentralised solution, which
strategy do you plan to use?  
\end{question}

The solution follows a decentralised architecture in which agents run completely
decoupled in different processes. Agents share no memory and decision-making
takes place individually, even though every agent communicates part of his
perception to the others.

\begin{question}
What is the communication strategy and how complex is it?
\end{question}

Percepts are communicated among agent members of the team via a broadcast
mechanism developed as part of the multi-agent system. This design was chosen
for its minimal complexity.

\begin{question}
How are the following agent features considered/implemented: autonomy,
proactiveness, reactiveness?  
\end{question}

Agents are completely autonomous; decision-making takes place individually at
the agent level, with no intervention from human operators or a central
intelligence agency within the system. Agents assign priorities to different
possible goals depending on their desires, and plan in order to achieve the most
valuable goal. This results in more autonomous way in which an agent acts.  The
agents' behaviour can be considered proactive, given they pursue their selected
intentions over time, that is, they have persistent goals.

\begin{question}
Is the team a truly multi-agent system or rather a centralised system in
disguise?  
\end{question}

The team is a truly multi-agent system, and has no centralised characteristics
beyond the sharing of all percepts among the team agents.

\begin{question}
How much time (man hours) have you invested (approximately) for implementing
your team?  
\end{question}

% 28 weeks, 8 hs per week, 6 developers ~= 1500

About 1500 hs.

\begin{question}
Did you discuss the design and strategies of your agent team with other
developers? To which extent did your test your agents playing with other teams?
\end{question}

Experience from a previous instance of the MAPC was shared with our teams by
members of the ARGONAUTS team from TU Dortmund\cite{Holzgen:2011}.  Although the
initial plan was to run tests against other agent teams prior to the
competition, time constraints made this impossible.

\subsection{Software Architecture}
% == Software Architecture 
\setcounter{question}{0}

\begin{question}
Which programming language did you use to implement the multi-agent system?  
\end{question}

The agent system was implemented using Python 2.7 and SWI Prolog 5.10.5. DeLP,
a defeasible logic language, was used as a service within Prolog.

\begin{question}
Did you use multi-agent programming languages? Why or why not to use
a multi-agent programming language?  
\end{question}

No multi-agent programming languages/platforms/frameworks were used. Being the
first time we participated in the contest, we decided not to use technologies
that we had absolutely no experience on. Besides, one of our goals was to
develop our own platform in order to keep developing in the future.

\begin{question}
How have you mapped the designed architecture (both multi-agent and individual
agent architectures) to programming codes i.e., how did you implement specific
agent-oriented concepts and designed artifacts using the programming language?  
\end{question}

The perception is processed by the Python program, that parses the XML. Then, it
sends it to the Percept Server that every step merges all perceptions, and
delivers them back to the agents.  The Python code asserts all the perception
into Prolog, then querying it for the next action to be executed.  Prolog
handles all the decision making, argumentation and planning, and returns the
action binded to a variable to Python, that then generates with it an XML to be
sent to the server.

\begin{question}
Which development platforms and tools are used? How much time did you invest in
learning those?  
\end{question}

All our code was written using either vim (on Linux) or Notepad++ (on Windows).
We used no IDEs, but occasionally we did use the SWI-Prolog integrated debugger.

\begin{question}
Which runtime platforms and tools (e.g. Jade, AgentScape, simply Java, ....) are
used? 
\end{question}

How much time did you invest in learning those?  Python and Prolog were the
chosen languages for the development of the system. Most of us had already
worked with both of them, so we did not spend much time learning those.

\begin{question}
What features were missing in your language choice that would have facilitated
your development task?
\end{question}

\begin{question}
What features of your programming language has simplified your development
task?  
\end{question}

Python's amenity to rapid application development and 'batteries-included
philosophy' facilitated implementing the communication layer to the MASSim
server, parsing of peceptions, rapid addition of planned features and bug
correction.  We made use of Prolog's declarative nature to model states of the
world, and it also made it more straightforward to implement search algorithms.

\begin{question}
Which algorithms are used/implemented?  
\end{question}

Search algorithms, as Uniform Cost Search and Depth First Search, as well as the
zone-coloring algorithm were implemented in Prolog.  The implementation of
Defeasible Logic Programming (DeLP) by the LIDIA was used for the deliberative
process.

\begin{question}
How did you distribute the agents on several machines? And if you did not
please justify why.  
\end{question}

Initial plans were to distribute agents on several machines. Each agents runs as
a separate process, and communicates with others via TCP sockets. After some
experience and benchmarking, agents were run on one machine due to performance
issues. Having the choice was a benefit of the proposed design.

\begin{question}
To which extend is the reasoning of your agents synchronized with the
receive-percepts/send-action cycle?
\end{question}

All the reasoning is done after receiving the percepts, and before sending the
action.

\begin{question}
What part of the development was most difficult/complex? What kind of problems
have you found and how are they solved?  
\end{question}

The most difficult problems were related to optimization. Much of our time has
been spent in reducing the complexity of our algorithms, and the times they are
called.

\begin{question}
How many lines of code did you write for your software?  
\end{question}

Total LOC is 5842.

% == Strategies, Details, and Statistics
\subsection{Strategies, Details, and Statistics}
\setcounter{question}{0}

\begin{question}
What is the main strategy of your team?
\end{question}

The main strategy of the team consists of detecting profitable zones from the
explored vertices, and positioning the agents correctly to maintain, defend and
expand the zones.  Every agent, beyond its role, is concerned with the formation
and expansion of zones.  The desicion-taking process is responsible for
calculating and selecting the most beneficial intention, which may be focused in
the zone conquering (if possible), or not.
    
\begin{question}
How does the overall team work together? (coordination, information sharing,
...) 
\end{question}

The agents coordinate in an implicit way. This is, the  information shared
consists only of the perception received, without having neither preprocessed
beliefs, nor control variables. The agents do not communicate their intention,
or plans, so any coordination that they may have has been achieved implicitly.

\begin{question}
How do your agents analyze the topology of the map? And how do they exploit
their findings? 
\end{question}

Agents make no assumption about the map topology. The exploration of the map is
done gradually, as a result of the reasoning proccess.

\begin{question}
How do your agents communicate with the server?  
\end{question}

Some functionality provided by the eismassim library was reimplemented in
a connection library in Python.

\begin{question}
How do you implement the roles of the agents? Which strategies do the different
roles implement?  
\end{question}

Agents recover their assigned role from the simulation start message.  Each role
has a couple of separate files, that have specific code, including the arguments
used in the decision-taking module, and the setting of the beliefs needed for
those arguments.  All agents follow the same concept. Every agent is concerned
with the formation and expansion of zones, beyond its role.

\begin{question}
How do you find good zones? How do you estimate the value of zones?  
\end{question}

Agents are not primarily focused on finding new zones, but they attempt to
expand and maximize the points of the existing ones. They calculate whether they
are part of a zone or not.  If an agent is not being part of any zone, it tries
to regroup with its teammates. When a zone is formed, and the agent is part of
it, for each potentially beneficial neighbor node, the agent calculates how much
points would the team gain if it moves, and tries to expand the zone.  This
estimations are done with our reimplementation of the coloring algorithm used by
the MASSim server.

\begin{question}
How do you conquer zones? How do you defend zones if attacked? Do you attack
zones?  
\end{question}

Both attacking and defense of zones are implicitly implemented.  Sabouteurs
prefer to attack enemies that are near, so if an agent of another team enters
our team's zone, it will be attacked by the saboteurs in the zone.  The same
happens with enemies in their own zones. Zones are not intentionally destroyed,
but any agent that is part of a zone may be attacked, affecting posibbly the
structure of the enemy zone.

\begin{question}
Can your agents change their behavior during runtime? If so, what triggers
the changes?  
\end{question}

Our agents do not change their behavior during runtime. This feature was
analysed, but the team did not have enough time to finish its implementation.

\begin{question}
What algorithm(s) do you use for agent path planning?  
\end{question}

Path planning is implemented with an Uniform Cost Search. What we tried to
minimize was the amount of steps required to achieve the goal, rather than the
spent energy. The returned result is a list of actions to be done.

\begin{question}
How do you make use of the buying-mechanism?  
\end{question}

Agents follow a list of predefined buying actions, when the necessary amount of
money is reached. This behaviour follows the idea of getting some specific skill
upgrades that the team considered important to achieve early in the simulations.

\begin{question}
How important are achievements for your overall strategy?  
\end{question}

Achievements are not explicitly taken under consideration. That is, the agents'
reasoning proccess is not affected by the possibility of completing
achievements.    However, the team can manage to achieve a significant number of
them, which results naturally from the agents' behaviour. 

\begin{question}
Do your agents have an explicit mental state?
\end{question}

Agents have a complete and explicit mental state. It consists of a set of
components, such as beliefs, desires, intentions, and plans. 

\begin{question}
How do your agents communicate? And what do they communicate?  
\end{question}

Agents only communicate their perceptions via a perception server implemented in
Python.  On each perceive/act cycle, agents receive the percept from the MASSim
server, separate the information which will remain private and which will be
shared.  The public part of the percept is sent to the percept server, which
performs a union of all percept and send the difference back to each agent.
After receiving the joint percept, the agents enter a belief setting phase, and
later an argumentation phase.

\begin{question}
How do you organize your agents? Do you use e.g. hierarchies? Is your
organization implicit or explicit?
\end{question}

There is no agent hierarchy, and given the decision-making process takes place
individually for each agent, there is no organization between them.  The only
organization that they have is the proper given by the environment, which is the
roles.

\begin{question}
Is most of your agents behavior emergent on an individual and team level?
\end{question}

The desicion-taking module makes use of other agents' status, but there is
neither negotiation nor intentions exchange, so the team performance is emergent
on an individual behaviour.

\begin{question}
If your agents perform some planning, how many steps do they plan ahead?
\end{question}

The agents make plans as long as the selected intention requieres. This may
sound excesive, but the possible goals were previously selected for their
potential, taking in consideration their distance (in nodes, not in actions).
However, plans are recalculated in every step.

% == Conclusion
\subsection{Conclusion}
\setcounter{question}{0}

\begin{question}
What have you learned from the participation in the contest?
\end{question}

Being our first experience building a system this size, we learned several
lessons about working in big projects, such as setting standards and
synchronizing versions of the technologies used.

\begin{question}
Which are the strong and weak points of the team?
\end{question}

Our team is formed by a group of friends, so a strong point in the developing
proccess was the cohesion and comradeship. We had already worked together, both
in university's projects and in small freelance projects, so we well knew what
to expect from each other, and each member's capabilities.

\begin{question}  
How suitable was the chosen programming language, methodology, tools, and
algorithms?
\end{question}

We are really happy with our decisions involving programming languages, tools
and algorithms. Of course, we had many problems, and much time was spent
deciding what to choose, but having all the proccess in consideration, we may
had took the right calls.

\begin{question}
What can be improved in the context for next year?
\end{question}

There were several hotfixes that were written and deployed at the same time we
were facing our competitors due to the lack of testing in the actual context of
the competition. This situation should obviously not happen, and adding much
more real testing is one of our main priorities for next year's competition.

\begin{question}
Why did your team perform as it did? Why did the other teams perform
better/worse than you did?
\end{question}

We had several problems that didn't let us perform as good as we expected.  Our
lack of experience in this kind of contests, unexpected network and latency
problems, as well as some bugs that caused critical performance issues, caused
our team to lose several matches that could've been won otherwise.

\begin{question}
Which other research fields might be interested in the Multi-Agent
Programming Contest?
\end{question}

Both Robotics and Gaming AI are interesting fields that could benefit from
participating in the contest.

\begin{question}
How can the current scenario be optimized? How would those optimization pay off?
\end{question}

More information for the nodes, including something useful for a directed search
(i.e., absolute coordinates), would help in the implementation of a A* search
(which would decrease execution time).  Defining the most valuable zones
randomly would benefit teams that thoughtfully look for and conquer good zones,
rather than teams that assume that the center of the map is the most valuable
zone and don't explore the rest.

A more informing/informational/apprising? feedback from the server would be
appreciated, specially involving errors.

Finally, we think it will be really helpful for all that we had test matches in
a more early stage, in order to have more time to correct errors in the client.
